%---------------------------------------------------------------------------------------------------
% Zusammenfassung
%---------------------------------------------------------------------------------------------------
% \newpage
%%\part{Schluss}
\chapter{Fazit}


\begin{flushleft}
In der Einleitung dieser Hausarbeit wurde darauf hingewiesen, dass die F�lle an Vielfalt sich nur entfalten kann, wenn allen die Chance an gesellschaftlicher Teilhabe und Teilnahme erm�glichen wird. Der Grundstein hierf�r sind gleiche Bildungschancen f�r alle. Doch wie ist es realisierbar?
\end{flushleft}


\begin{flushleft}
Die Hauptzielsetzung dieser Arbeit war es den Unterschied von Integration und Inklusion zu verdeutlichen. Dazu erfolgte zu erst die Beleuchtung des geschichtlichen Hintergrundes, sowie die Aufzeigung der Unterschiede zwischen Integration und Inklusion und welche Ver�nderungen es f�r das deutsche Bildungssystem bedeutet. Dem folgte die rechtliche und gesetzliche Auseinandersetzung mit dem Thema Inklusion, dies geschah sowohl auf der nationalen als auch auf internationalen Ebene. Zudem wurde der Begriff Heterogenit�t definiert und der Umgang mit Vielfalt wurde weiter er�rtert. Abschlie�end wurde auf die Frage eingegangen wie sich inklusive P�dagogik in der Praxis umzusetzen l�sst. 
\end{flushleft}

\begin{flushleft}
Durch die Hausarbeit k�nnte gezeigt werden, dass der Grundgedanke von Inklusion- die Einbeziehung und teilhabe aller- nicht neue ist. Den Grundstein legte in Deutschland, vor �ber vier Jahrzehnten, das Integrationsbestreben von Eltern. Sie k�mpften daf�r, dass ihre Kinder mit besonderen F�rderbedarf Regelkindertagesst�tten besuchen durften. Doch im Laufe der Jahre wurde immer deutlicher, dass das deutsche Bildungssystem nach mehr Ver�nderungen verlangt. Im Zuge dessen wurde der Integration Begriff durch den Begriff Inklusion abgel�st bzw. weitergef�hrt.
\end{flushleft}

\newpage
\begin{flushleft}
Erste Schritte im Entwicklungsprozess f�r eine inklusive P�dagogik wurden bereits vollzogen, doch sind wir lange noch nicht am Ziel angekommen. Zur Zeit wird zwar an vielen Kn�pfen gedreht um diesem Ziel einwenig n�her zu kommen, doch verlangt Inklusion mehr als neue politische Programme und Gesetze. Es gilt eine Haltung zu entwickeln und zu verinnerlichen, die Vielfalt als Chance und Ressource f�r Erziehung und Bildung von Kindern sieht. Hierf�r braucht es mehr Unterst�tzung von den Verantwortlichen, den Inklusion gibt es nicht zum Nulltarif. Au�erdem sollen Bedingungen geschaffen werden die es allen Beteiligten einfacher macht, dieses komplexe Thema anzugehen. Dem zu Folge sollte Inklusion, damit sie gelingt, nicht an der Kita T�r enden.
\newline\newline
\enquote{\emph{Wir m�ssen selbst die Ver�nderung sein, die wir in der Welt sehen wollen --- Mahatma Gandhi}}
% [Vielfalt2012, 73]
\end{flushleft}

\begin{flushleft}
Die Frage, die mich am Ende dieser Arbeit noch interessiert ist - Inwieweit muss sich die Ausbildung von fr�hp�dagogischen Fachkr�ften erweitern bzw. ver�ndern, damit sie allen Anforderungen, die inklusiver P�dagogik an das fr�hp�dagogische Fachpersonal richtet, gerecht werden kann? Die Auseinandersetzung mit dieser Frage, w�rde den Umfang dieser Arbeit �berschreiten. Es empfiehlt sich daher, dieses Thema in einer weiteren wissenschaftlichen Arbeit zu betrachten.
\end{flushleft}
