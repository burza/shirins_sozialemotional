%---------------------------------------------------------------------------------------------------
% Einf�hrung
%---------------------------------------------------------------------------------------------------
% \newpage
%%\part{Anfang}
\chapter{Einleitung}

\begin{flushleft}
Vielfalt von Anfang an-  das Deutsche Bildungssystem ist durch Homogenit�t \footnote{Bezeichnet die Gleichheit einer physikalischen Eigenschaft �ber die gesamte Ausdehnung eines Systems., url: \url{http://de.wikipedia.org/wiki/Homogenit�t}, gesehen am: 29.08.2012 12:30} gepr�gt. Folglich werden, um m�glichst homogene Gruppen zu bekommen, Kinder mit besonderen Bed�rfnissen in Sondereinrichtungen unterst�tzt. Allerdings ist durch �ffentliche Diskussionen �ber Chancengleichheit die Fach�ffentlichkeit zu der Erkenntnis gekommen, dass dies nur gelingt, wenn alle Menschen die M�glichkeit auf gesellschaftliche Teilhabe bekommen. Um die Chance f�r Teilhabe zu erm�glichen, bedarf es an gleichen Bildungschancen f�r alle. So wird die Fr�hkindliche Bildung  als Fundament f�r eine gelingende Bildung gesehen. Die Hoffnung der Bundesregierung ist, dass herkunftsbedingte Benachteiligungen durch eine fr�he Bildung minimiert und kompensiert werden k�nnen, da besonders Kinder mit Behinderungen, ebenso wie Kinder, die von Armut betroffen sind und Kinder mit Migrationshintergrund  ein erh�htes Risiko haben keine faire Chance auf gesellschaftliche Teilhabe zu bekommen. Vorreiter f�r eine Ver�nderung im Bildungssystem ist die Fr�hp�dagogik, die sich schon seit vier Jahrzehnten an einer integrativen Bildung und Erziehung orientiert. Doch durch die Einf�hrung eines neuen  Begriffes, Inklusion, der die Neubetrachtung des Begriffes Integration verlangt, aber auch als Weiterf�hrung von Integration verstanden wird, sind neue Debatten in der Fach�ffentlichkeit entfacht. Dies kann ein Wegbereiter f�r einen Wandel des deutschen Bildungssystems sein. 
\end{flushleft}

\newpage
\begin{flushleft}
Deutschland steht zwar noch am Anfang vom Inklusionsprozess, aber erste Schritte auf dem noch langen Weg sind gemacht.  Es stellt sich jedoch die Frage-  inwieweit sind die Menschen in Deutschland bereit f�r Inklusive P�dagogik? Diese Frage greift auch gleich den Titel und die Hypothese dieser Arbeit auf- Inklusive P�dagogik- Deutschland zwischen Integration und Inklusion. In dieser Arbeit m�chte ich nach einer Einf�hrung in das Thema Inklusion, in diesem Sinne den Unterschied zwischen Integration und Inklusion verdeutlichen. Hierbei wird deutlich, dass Inklusion noch einen Schritt weiter geht als Integration. Des Weiteren werde ich auf die Zusammenh�nge zum Thema Vielfalt in Kindertageseinrichtungen eingehen und dann weiterf�hren wie sich inklusive Bildung in der Praxis umsetzen l�sst. Im letzten Kapitel fasse ich zusammen und gebe mein Fazit wieder.
\end{flushleft}

\begin{flushleft}
Viele der Themen in dieser Arbeit konnte ich leider nur anrei�en, da es sonst den Umfang dieser Arbeit �berschritten h�tte. Trotzallem hoffe ich, dass die Hausarbeit dem Leser einen kleinen Einblick in das Thema Inklusion gibt. Des Weiteren m�chte ich darauf hinweisen, dass die Arbeit sich haupts�chlich auf  die fr�hkindliche P�dagogik bezieht.
\end{flushleft}



\newpage