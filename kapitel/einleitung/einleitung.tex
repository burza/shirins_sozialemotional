%---------------------------------------------------------------------------------------------------
% Einf�hrung
%---------------------------------------------------------------------------------------------------
% \newpage 
%%\part{Anfang}
\chapter{Einleitung}

\begin{flushleft}
Von Beginn an ist der Mensch ein emotionales und soziales Wesen. Die Gemeinschaft spielt eine entscheidende Rolle in der Entwicklung der eigenen Identit�t. In ihr werden uns st�ndig soziale und emotionale Erfahrungen geboten. Doch das Zusammenleben mit anderen gelingt nur, wenn der Mensch �ber sehr komplexe soziale und emotionale F�higkeiten verf�gt. Diese F�higkeiten ben�tigt er, um sich mit anderen verst�ndigen und sich auf ihre W�nsche und Bed�rfnisse einstellen zu k�nnen. Dabei ist es wichtig, dass die  eigenen Emotionen und Bed�rfnisse erkannt und ausgedr�ckt werden. Der Erwerb von sozial-emotionalen Kompetenzen ist ein bedeutender Entwicklungsschritt f�r Kinder. 
\end{flushleft}

\newpage
\begin{flushleft}
Auf Grund des aktuellen Diskurses in der Bildungslandschaft, ist die Bedeutung sozial-emotionaler Kompetenzen wieder vermehrt in den Vordergrund ger�ckt. Das h�ngt zum einen mit den Ver�nderungen in Umwelt und Gesellschaft, zum anderen h�ngt es mit den immer h�her werdenden Erwartungen und Anforderungen an Kinder zusammen. Des Weiteren wird das Kind jetzt mehr als selbstbildendes Wesen wahrgenommen. Um all den Herausforderungen unserer heutigen Gesellschaft gerecht werden zu k�nnen, werden von Heranwachsenden vielf�ltige Interaktions- und  Handlungsmuster abverlangt. Die Basis hierf�r wird in der fr�hen Kindheit gelegt. Hierbei bietet der p�dagogische Alltag, von Kindertageseinrichtungen, eine Vielfalt an M�glichkeiten. So k�nnen neben Alltagssituationen auch Spiele und Bewegungsangebote soziale Prozesse anregen und die sozialen und emotionalen Kompetenzen f�rdern. Die Aufgabe des p�dagogischen Fachpersonales ist es, anregende Situationen zu schaffen, in denen Kinder sich im sozialen Handeln ausprobieren k�nnen. Hierbei sollten sie eine Balance zwischen Anregung und Selbstbildung erm�glichen.
\end{flushleft}

\begin{flushleft}
Im Verlauf dieser Arbeit werde ich folgender Frage nachgehen:  \enquote{Welchen Chancen bieten Spiel und Bewegung hinsichtlich der F�rderung von soziale-emotionalen Kompetenzen?}. Am Anfang dieser Arbeit werde ich die Begriffe soziale und emotionale Kompetenzen jeweils n�her erl�utern. Hierbei wird deutlich,wie sehr sich diese beiden Kompetenzen bedingen. Des Weiteren werde ich auf den Erwerb von soziale-emotionalen Kompetenzen durch spielen und bewegen eingehen und daraufhin weiterf�hren, wie es sich in der Praxis umsetzen l�sst. Au�erdem gebe ich einen kleinen Einblick in das Projekt SEKIP, das zurzeit ebenfalls der Frage, der vorliegenden Arbeit nach geht. Im letzten Kapitel fasse ich zusammen und gebe mein Fazit wieder.
\end{flushleft}

\begin{flushleft}
Viele der Themen konnte ich nur anrei�en, da es sonst den Umfang dieser Arbeit �berschritten h�tte. Trotz allem hoffe ich, dass diese Hausarbeit dem Leser einen kleinen Einblick in die Chancen die Bewegung und Spiel f�r die F�rderung sozial-emotionale Kompetenzen bietet geben wird. 
\end{flushleft}

\begin{flushleft}
Des Weiteren m�chte ich darauf hinweisen, dass die Arbeit sich haupts�chlich auf  die fr�hkindliche P�dagogik bezieht. Au�erdem spreche in dieser Arbeit, der Einfachheit halber, vom p�dagogischen Fachpersonal.
\end{flushleft}

\newpage